\begin{flushleft}
    \huge
    \textbf{5. Evaluation}
    \vspace{0.1cm}

    \large
    \begin{enumerate}
        \item {\Large Evaluation of Objectives} \\
            \vspace{0.2cm}
            In this section, I will evaluate all of my objectives I set out to complete.
            \vspace{0.2cm}

            \begin{enumerate}
                \item Reading user inputted data \\
                    \vspace{0.2cm}
                    The user can input the parameters through a json file, and these parameters are checked against a range file to check they are within
                    the specified size. All of the parameters are read correctly and utilised within the Program. \\
                    \vspace{0.2cm}
                    The Machine Learning Data is read from .dqn files. The Learning is resumed from where it was saved from with all the Weights and
                    Biases intact. \\

                    \vspace{0.5cm}    
                \item Generating the Environment \\
                    \vspace{0.2cm}
                    At the start of the program an instance of World Class is created and the Generate methods are invoked. These methods utilise Perlin
                    Noise and Poisson Disc Sampling. The Terrain values are stored in a 2d list of Tile Objects which store the Height, Type and Colour data
                    for each Tile. The Poisson Disc Sampling Generates a list of points which Trees are then generated at those positions. The Width
                    of the world and Tile colours are determined by the Input Parameters. \\

                    \vspace{0.5cm}    
                \item Displaying the world to a Pygame Window \\
                    \vspace{0.2cm}
                    Upon generating the Map Data the Terrain is displayed in a grid to the Pygame Window, it is represented as a grid of tiles of the pixel 
                    width loaded in by the Inputted Parameters. The Agent and Enemies are Drawn at their according positions, taking up entire Tile. If Debug
                    mode is enabled, a representation of the Neural Network will be displayed on the right hand side of the window. \\

                    \vspace{0.5cm}   
                \item Simple Agent with a set of Actions \\
                    \vspace{0.2cm}
                    An Agent can be created as an object and works along side the Dual Neural Network Object to enable interactions between the environment and
                    the Network. The Agent can collect the surroundng Tile Data using the \textbf{GetTileVector} Method, this can then be converted into the
                    Networks Input Vector using the \textbf{TileVectorPostProcess} Method. There exists Methods to Take a given Action, normally outputted by
                    the Network. Along with Methods to Calculate Reward for an Action given a State, or the Maximum Possible Reward Given a State. \\
                    \vspace{0.2cm}
                    There also exists Methods to Reset the Agent to its default values. Along with Determining the Agents Spawn Position when given a WorldMap
                    Object. \\

                    \vspace{0.5cm}   
                \item Matrix class with Standard Operations \\
                    \vspace{0.2cm}
                    A Matrix can be created using 3 different methods. First using a Tuple of Integers, a new Matrix will be created of that size, with initialised
                    0 values. Second using a prexisting 2d list of values, a new Matrix will be created with these dimensions and values. Thirdly a 1d list of
                    values can be used to create a 1 wide Vector of values, where it reads each value into the 1st position of each row. \\
                    \vspace{0.2cm}
                    All standard operations for the Matrix Object are implemented using Operator Overloading to make code less bloated. All are written 
                    efficiently utilising minimum complexity algorithms. Addition can be carried out utilising the $+$ Operator. Subtraction can be carried out
                    utilising the $-$ Operator. Multiplication and Scalar Multiplication are both carried out utilising the $*$ Operator. Power Operation is
                    carried out utilising the $\string^$ operator. A Matrix can be converted to a Formatted String implicitly by using it in a string context. \\
                    \vspace{0.2cm}
                    All Matrice Operations have appropriate Exceptions with descriptive Error Messages. They will throw errors when incorrect Data is provided to
                    the specified Operation. \\

                    \vspace{0.5cm}   
                \item Creation of a Reinforcement Learning Model \\
                    \vspace{0.2cm}
                    A Dual Neural Network can be created as an object, which stores two Neural Network Objects, Main and Target. The Dual Neural Network
                    contains the Primary Method \textbf{Step} which invokes a Series of Lower Level Methods to perform a singular Time Step. The Neural 
                    Network Object store a List of Layers Objects which are dynamically created from the Input Parameters. Each Layer contains a Weight 
                    Matrix, Bias Vector, and Output Vector. The Lowest Level methods for Forward and Back Propagation are contained within the Layer Object. \\
                    \vspace{0.2cm}
                    First Forward Propagation occurs on the Main and Target Network. Then results of the Main Network are taken to choose the action for the Agent. 
                    Epsilon Greedy is implemented to determine whether to choose the random or predicted result. This Action is then fed to the Agent, along with 
                    calculating the reward for that Action. The Loss of the Main Network is then calculated using a modified Bellman Equation for Dual Neural 
                    Networks. This Loss is used for Back Propagating the Main Neural Network. The Main Networks Weights are copied to the Target Network
                    every specified ammount of steps. Every specified ammount of steps, Experience Replay is performed to learn from past experiences again. \\
                    \vspace{0.2cm}
                    The combination of these steps form a functional Dual Neural Network utilising a Reinforcement Learning Model. \\

                    \vspace{0.5cm}   
                \item Creation of a Data Logger \\
                    \vspace{0.2cm}
                    A Data Logger Class can be used to Log and Store Data Points at various parts of the Program. Each Data Point is stored as a Tuple of Values
                    as part of a .data file. These files are stored as Binary Files, and are Read into the Program upon launch. \\
                    \vspace{0.2cm}
                    As part of the Data Logger you can sort points utilising a Heap Sort to sort through Data. \\

                    \vspace{0.5cm}   
            \end{enumerate}

            \vspace{0.5cm}
        \item {\Large Answering my Investigations Question} \\
            \vspace{0.2cm}
            As part of my Machine Learning Investigation I proposed the Question:

            \vspace{0.3cm}\begin{center}
            \textbf{Can you train a Machine Learning algorithm to survive in a pseudo random, open-world environment?}
            \end{center}\vspace{0.3cm}

            I aimed to answer this question by designing and creating a Deep Reinforcement Learning Model utilising a Deep Neural Network, along 
            with designing a Simple Simulation for a Machine Learning Agent to survive in. This simulation

            \vspace{0.5cm}

        \item {\Large Expert Feedback} \\
            \vspace{0.2cm}
            I went back to my Expert Shaun in order to collect feedback on my finalised Technical Solution. I asked him a few Questions about my
            project, paraphrased where neccesary. \\
            \vspace{0.5cm}

            \begin{enumerate}
                \item What do you think of the Program? \\
                    \vspace{0.2cm}
                    "Overall I think your project is incredibly visually interesting to look at, I could stare at the graphical output for hours
                    just rooting for the Agent to better itself and kill the Generated Enemies. The User Inputted Parameters are easy to change
                    through the json file, and it is helpful that they are locked between certain ranges to stop the User from crashing their Pc
                    from allocating too much memory. The Terrain generation looks pretty good for just a 4 coloured map generated from Perlin Noise.
                    The Neural Network works as intended, although \textbf{NOT FINISHED}"

                \item Does my Tehnical Solution achieve all of the Set Goals and Objectives? \\
                    \vspace{0.2cm}
                    "The Program achieves all of the objectives you set out to complete, and it is clear alot of hard work went into completing your
                    project. Lots of research needs to be carried out in order to understand the complexity behind Reinforcement Learning and all
                    of its individual parts. Debugging this process also becomes increasingly difficult, due to the complex calculations, this 
                    demonstrates you have the ability to solve problems independently. \\
                    \vspace{0.2cm}
                    You've also implemented an entire simulation ontop of the Dual Neural Network. Which uses even more complex algorithms, this demonstrates
                    you can develop multiple Vertical Slices of a project, and intertwine them together in order to create one bigger project. This
                    takes planning skill and a good understanding of OOP in order to pull off." \\

                    \vspace{0.5cm}
                \item What Criticisms/Improvements would you suggest? \\
                    \vspace{0.2cm}
                    "Considering the scope of the project, youve carried out your completion of this task very well. The only suggestion I would have is
                    to implement a Convolution, which might solve your Training Accuracy Problems. Otherwise a Description of your Project could be
                    printed to console when the main file is run, or a 'Readme' text file included in the project files would useful to any users who 
                    have little to no experience with Reinforcement Learning." \\

                    \vspace{0.5cm}
            \end{enumerate}


            \vspace{0.5cm}
        \item {\Large Evaluation of Expert Feedback} \\
            \vspace{0.2cm}
            

            \vspace{0.5cm}

        \item {\Large System Improvements} \\
            \vspace{0.2cm}
            Overall I am happy with my Technical Solution. I achieved all the objectives I set out to complete in my Analyis. I have definitely achieved
            my primary goal of gaining a deeper understanding about the Maths and Logic behind how Neural Networks work. This has given me a Window into
            the field of Machine Learning and Artificial Intelligence, which I intend to pursue as part of my later Studies. \\
            \vspace{0.2cm}
            The Improvements I would like to make to my Technical Solution are: \\
            \vspace{0.5cm}

            \begin{enumerate}
                \item The Implementation of a Convolutional Neural Network was something I came across in my Initial Research and was mentioned by my Expert.
                Convolution carries out Pre-Processing on the inputted data before it is even touched by the Neural Network. This in theory would increase
                the training accuracy of my Network leading to better Results.
                
                \item The Optimisation of my Matrix Class by compiling it into $C$ through the use of Cython would help speed up the training of the Neural
                Network. Due to Python being an interpretted language it is comparatively slow compared to the other programming languages I considered
                using. $C$ is a compiled language so it is comparatively alot faster, about 45 times faster according to some sources online. This could
                provide an easy way to optimise my Program without having to convert my entire Codebase into a different Language.

                \item An increase in complexity of my simulation would provide a greater challenge towards my Agent and Neural Network. I could add a basic
                crafting system to convert the collected Wood into a sword, or a Hunger Bar so the Agent has to collect food and water in order to survive.
                I feel as though the Network wouldnt be able to solve these problems effectively though without the implementation of my first improvement,
                a Convolutional Neural Network. 
            \end{enumerate}}

            \vspace{0.5cm}

    \end{enumerate}
\end{flushleft}