\begin{flushleft}
    \huge
    \textbf{3. Design}

    \Large
    \begin{enumerate}
        \item {\Large Whole System Flow Chart}
            \large
            \vspace{0.2cm}

        \item {\Large System Class Diagrams}
            \large
            \vspace{0.2cm}

        \item {\Large Description of Algorithms}
            \large
            \vspace{0.2cm}
            \begin{enumerate}[label=\arabic*)]
                \item Matrix Addition \\
                This algorithm is a Mathematical Operation to add 2 Matrices together. To Add together 2 Matrices their Orders
                must be the same. To perform the Operation you must Sum each element in Matrix A with the corresponding element 
                in Matrix B, placing the result of each Sum in the resultant Matrix.

                \vspace{0.5cm}
                \item Matrix Subtraction \\
                This algorithm is a Mathematical Operation to subtract 2 Matrices. To Subtract 2 Matrices their Orders
                must be the same. To perform the Operation you must Sum each element in Matrix A with the negative of the 
                corresponding element in Matrix B, placing the result of each Sum in the resultant Matrix.

                \vspace{0.5cm}
                \item Matrix Multiplication \\
                This algorithm is a Mathematical Operation to find the product of 2 Matrices. To Multiply 2 Matrices
                the number of Columns in the Matrix A must be equal to the number of Rows in Matrix B. Where Matrix A has
                dimensions of $m$ x $n$ and Matrix B has dimensions of $j$ x $k$, the resultant Matrix will have dimensions of 
                $n$ x $j$. To Multiply two Matrices, the algorithm performs the Dot Product between the Row in Matrix A and the 
                corresponding Column in Matrix B. The Dot Product is the Sum of the Products of corresponding elements.

                \vspace{0.5cm}
                \item Matrix Scalar Multiplication \\
                This algorithm is a Mathematical Operation to find the product between a Matrix and a Scalar.
                The result can be found by Multiplying each element of the Matrix by the Scalar Value to form the Resultant 
                Matrix.
                
                \vspace{0.5cm}
                \item Matrix Hadamard Product \\
                This algorithm is a Mathematical Operation to another way to find the product between 2 Matrices. Instead of
                applying the Dot Product between Rows and Columns, you find the product between each element in Matrix A
                with the corresponding element in Matrix B, placing the result in the resultant Matrix. This is very epic gamer

                \vspace{0.5cm}
                \item Matrix Power \\
                This algorithm is a Mathematical Operation to find the power of a Matrix. The given Matrix needs to have square dimensions.
                The result can be found by multiplying the given Matrix by itself $n$ ammount of times where $n$ is the given power.
                
                \vspace{0.5cm}
                \item Matrix Transpose \\
                This algorithm is a Mathematical Operation used to Flip a Matrix across its Diagonal. The Transpose of any Matrix
                can be found by converting each Row of the Matrix into a Column. An $m$ x $n$ Matrix will turn into an $n$ x $m$ Matrix.
                
                \vspace{0.5cm}
                \item Activation Function Sigmoid \\
                This algorithm is a Mathematical Formulae which squishes any value to between 0 and 1. It uses Eulers Number $e$.

                \vspace{0.2cm}
                {\Large\centerline{$S(x) = \frac{1}{1 + e^{-x}}$}}
                \vspace{0.2cm}
                
                \vspace{0.5cm}
                \item Activation Function TanH \\
                This algorithm is a Hyperbolic Function which squishes any value to between -1 and 1. It is a Ratio between the two Hyperbolic 
                functions SinH and CosH.

                \vspace{0.2cm}
                {\Large\centerline{$TanH(x) = \frac{SinH(x)}{CosH(x)} = \frac{e^x - e^{-x}}{e^x + e^{-x}}$}}
                \vspace{0.2cm}

                \vspace{0.5cm}
                \item Activation Function Relu \\
                This algorithm is a simple function which removes any negative values from its input, returning 0 instead.
                
                \vspace{0.5cm}
                \item Activation Function SoftMax \\
                This algorithm is a logistic function that creates a probability distribution from a set of points. This probability 
                distribution sums to 1. It applies the standard Exponential Function to each element, then normalises this value by dividing
                by the sum of all these Exponentials.

                \vspace{0.5cm}
                \item Neural Network Forward Propagation \\
                This algorithm is used to obtain the outputs of a Neural Network. It uses Matrix Multiplication to propagate the inputs
                of the network from Layer to Layer, eventually reaching the Output Layer. My Multiplying the Weight Matrix and the outputs
                of the previous Layer, and then adding the Bias. We can obtain the output of the layer.
                
                \vspace{0.5cm}
                \item Neural Network Loss Function \\
                
                \vspace{0.5cm}
                \item Neural Network Back Propagation \\
                
                \vspace{0.5cm}
                \item Agent Get Tile Vector \\
                This algorithm takes the current world data of the simulation, and produces a Vector of Tile Data surrounding the Agent. 
                
                \vspace{0.5cm}
                \item Agent Post Process Tile Vector \\
                This algorithm will convert the Tile Vector into a Vector of Grayscale values, which can be used as the input for the Neural
                Network.
                
                \vspace{0.5cm}
                \item Agent Convert to Grayscale \\
                This algorithm converts a given RGB Colour Value to the corresponding Gray Scale Value

                \vspace{0.5cm}
                \item Agent Spawn Position \\
                
                \vspace{0.5cm}
                \item Enemy Spawn Position \\
                
                \vspace{0.5cm}
                \item Enemy Move \\
                
                \vspace{0.5cm}
                \item Poisson Disc Sampling \\
                
                \vspace{0.5cm}
                \item Perlin Noise \\
                
                \vspace{0.5cm}
                \item Octave Perlin Noise \\
                
                \vspace{0.5cm}
                \item Deque Push Front \\
                
                \vspace{0.5cm}
                \item Deque First \\
                
                \vspace{0.5cm}
                \item Deque Last \\
                
                \vspace{0.5cm}
                \item Deque Sample \\
                
                \vspace{0.5cm}
            \end{enumerate}

        \item {\Large Description of Data Structures}
            \large
            \vspace{0.2cm}

    \end{enumerate}
    \vspace{0.1cm}
\end{flushleft}